\paragraph{\textcolor{red}{Acknowledgment:} Our team members are \textbf{Alexander Lin (al1655)}, \textbf{Pranav Tikkawar (pt422)}, and \textbf{Ivan Zheng (iz72)}}.


\section*{Problem 1}
We covered the offline caching problem in class and stated that the farthest-in-future algorithm is optimal. Formally prove its correctness using the exchange argument. You can refer to our textbook [KT, Section 4.3] to see how the proof is done.

\paragraph{\textcolor{brown}{Solution.}}
Let S be an optimal strategy for the given request sequence. Let FiF be the strategy followed by the Farthest-in-Future algorithm. Assume that S and FiF behave identically up to a certain point where they differ in their eviction choices. We'll show that we can modify S to align with FiF at that point without increasing the number of misses. \begin{itemize} \item Consider the first time S and FiF differ in their eviction decisions. Let this be at request $r_i$ where item x is requested, and it's not in the cache. \item At this point, both S and FiF must evict an item to bring in x (since the cache is full and x is not present). \item Suppose FiF evicts item a (the one farthest in the future or never used again) and S evicts item b (b $\neq$ a). \item Since FiF chose a, this means a is used farther in the future than b, or a is not used again while b is used sooner. \item Define S' to be the same as S up to request $r_i$. \item At $r_i$, S' evicts a (like FiF) instead of b. \item After $r_i$, S' behaves optimally but ensures that it doesn't incur more misses than S. \item Case 1: a is never requested again. Evicting a is optimal since keeping b might lead to an earlier eviction of b later. S' doesn't incur extra misses compared to S. \item Case 2: a is requested later than b. By evicting a instead of b, b will be in the cache when it's next needed before a's next request. Thus, S' might avoid a miss on b that S would have had, or at least not do worse. \item In both cases, S' does not have more misses than S. \item Repeat this process for all points where S and FiF differ. Each exchange either keeps the number of misses the same or reduces them. Eventually, S is transformed into FiF without increasing the number of misses. \item Since any optimal strategy S can be transformed into FiF without increasing the number of misses, FiF must itself be optimal. \end{itemize}


\section*{Problem 2}
We have $n$ boxes. The box $i$ has weight $w_i$ and weight limit $\ell_i$. A box will break if the sum of weight above it exceeds its weight limit. Each box has the same size of $1 \times 1 \times 1$.

What is the tallest tower we can build by stacking a subset of boxes on top of each other? We can freely choose the order. Give a greedy algorithm and formally prove its correctness using the exchange argument.

\paragraph{\textcolor{brown}{Solution.}}
Here is my solution.


\section*{Problem 3}
Given a weighted undirected graph $G = (V, E)$, decide whether its minimum spanning tree is unique. Give an algorithm that runs in $O(m \log m)$ time and prove its correctness. [Hint: You can modify Kruskal's algorithm.]


\paragraph{\textcolor{brown}{Solution.}}
Here is my solution.


\section*{Problem 4}
In this problem, we will design an algorithm to compute the minimum spanning tree in a given graph in $O(m \log \log n)$ time, where $m$ is the number of edges and $n$ is the number of vertices in the given graph. For example, if $m = \Theta(n)$, then the running time of this algorithm ($\Theta(n \log \log n)$) is better than that ($\Theta(n \log n)$) of Prim's algorithm, Kruskal's algorithm, and Bor\r{u}vka's algorithm.
Here are the ideas.
\begin{itemize}
    \item In the original Bor\r{u}vka's algorithm, we might need $\Theta(\log n)$ rounds to merge all vertices into one group, and each round takes $O(m)$ time. The twist is that we now only run Bor\r{u}vka's algorithm for $r$ rounds.
    \item After running Bor\r{u}vka's algorithm for $r$ rounds, we obtain a new graph with at most $\frac{n}{2^r}$ vertices and at most $m$ edges. We run Prim's algorithm (with a Fibonacci heap) on it to find its MST.
\end{itemize}
How should we choose the parameter $r$ so that the total running time becomes $O(m \log \log n)$?

\paragraph{\textcolor{brown}{Solution.}}
Here is my solution.


\section*{Problem 5}
We have an undirected graph $G = (V, E)$. Each edge has a positive length, and there is at most one edge between each pair of vertices. Find the length of the shortest simple cycle in $G$. Your algorithm should run in $O(n^3)$ time, where $n = |V|$. [Hint: You can modify the Floyd--Warshall algorithm. Recall that $f(k, u, v)$ is the length of the shortest path from $u$ to $v$ using vertices with indices of at most $k$. Consider a cycle where $k$ is the largest index among its vertices and $u$ and $v$ are the two neighbors of $k$ in the cycle.]

\paragraph{\textcolor{brown}{Solution.}}
Here is my solution.

