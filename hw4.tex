\paragraph{\textcolor{red}{Acknowledgment:} Our team members are \textbf{Alexander Lin (al1655)}, \textbf{Pranav Tikkawar (pt422)}, and \textbf{Ivan Zheng (iz72)}}.


\section*{Problem 1}
We covered the offline caching problem in class and stated that the farthest-in-future algorithm is optimal. Formally prove its correctness using the exchange argument. You can refer to our textbook [KT, Section 4.3] to see how the proof is done.

\paragraph{\textcolor{brown}{Solution.}}
Let S be an optimal strategy for the given request sequence. Let FiF be the strategy followed by the Farthest-in-Future algorithm. Assume that S and FiF behave identically up to a certain point where they differ in their eviction choices. We'll show that we can modify S to align with FiF at that point without increasing the number of misses. \begin{itemize} \item Consider the first time S and FiF differ in their eviction decisions. Let this be at request $r_i$ where item x is requested, and it's not in the cache. \item At this point, both S and FiF must evict an item to bring in x (since the cache is full and x is not present). \item Suppose FiF evicts item a (the one farthest in the future or never used again) and S evicts item b (b $\neq$ a). \item Since FiF chose a, this means a is used farther in the future than b, or a is not used again while b is used sooner. \item Define S' to be the same as S up to request $r_i$. \item At $r_i$, S' evicts a (like FiF) instead of b. \item After $r_i$, S' behaves optimally but ensures that it doesn't incur more misses than S. \item Case 1: a is never requested again. Evicting a is optimal since keeping b might lead to an earlier eviction of b later. S' doesn't incur extra misses compared to S. \item Case 2: a is requested later than b. By evicting a instead of b, b will be in the cache when it's next needed before a's next request. Thus, S' might avoid a miss on b that S would have had, or at least not do worse. \item In both cases, S' does not have more misses than S. \item Repeat this process for all points where S and FiF differ. Each exchange either keeps the number of misses the same or reduces them. Eventually, S is transformed into FiF without increasing the number of misses. \item Since any optimal strategy S can be transformed into FiF without increasing the number of misses, FiF must itself be optimal. \end{itemize}


\section*{Problem 2}
We have $n$ boxes. The box $i$ has weight $w_i$ and weight limit $\ell_i$. A box will break if the sum of weight above it exceeds its weight limit. Each box has the same size of $1 \times 1 \times 1$.

What is the tallest tower we can build by stacking a subset of boxes on top of each other? We can freely choose the order. Give a greedy algorithm and formally prove its correctness using the exchange argument.

\paragraph{\textcolor{brown}{Solution.}}
We have $n$ boxes. The $i$-th box has weight $w_i$ and weight limit $\ell_i$. A box will break if the sum of weights above it exceeds its weight limit. Each box has the same size of $1 \times 1 \times 1$. The goal is to determine the tallest tower we can build by stacking a subset of boxes on top of each other, such that no box breaks. We can freely choose the order in which the boxes are stacked.\\
To solve this problem, we use a greedy algorithm based on sorting boxes by the sum of their weight limit and weight $(\ell_i + w_i)$ in non-increasing order. The algorithm proceeds as follows: \begin{enumerate} \item \textbf{Sort all boxes} in non-increasing order of $(\ell_i + w_i)$. \item \textbf{Process boxes in reverse order} (from last to first in the sorted list): \begin{itemize} \item Maintain a cumulative weight counter initialized to $0$. \item For each box, if its weight limit $\ell_i$ is greater than or equal to the cumulative weight, include it in the tower and add its weight $w_i$ to the cumulative counter. \end{itemize} \end{enumerate} The total number of boxes included in the tower is the height of the tallest possible tower. \\
We prove the correctness of the greedy algorithm using an exchange argument. Assume an optimal solution $O$ exists that differs from the greedy solution $G$. If there are two adjacent boxes $i$ and $j$ in $O$, where $i$ is below $j$, and $(\ell_i + w_i) < (\ell_j + w_j)$, swapping them preserves validity: \begin{itemize} \item In $O$, $\ell_i \geq w_j + S$ and $\ell_j \geq S$, where $S$ is the cumulative weight above box $j$. \item After swapping, $\ell_j \geq w_i + S$ and $\ell_i \geq S$. Since $(\ell_i + w_i) < (\ell_j + w_j)$, it follows that $\ell_j - w_i > \ell_i - w_j \geq S$, ensuring validity. \end{itemize} By iteratively swapping such pairs, $O$ can be transformed into $G$ without reducing the height of the tower. This proves that $G$ is optimal.



\section*{Problem 3}
Given a weighted undirected graph $G = (V, E)$, decide whether its minimum spanning tree is unique. Give an algorithm that runs in $O(m \log m)$ time and prove its correctness. [Hint: You can modify Kruskal's algorithm.]


\paragraph{\textcolor{brown}{Solution.}}
The algorithm to determine whether the minimum spanning tree (MST) of a weighted undirected graph \( G = (V, E) \) is unique involves the following steps:
\begin{enumerate}
    \item Compute the initial MST \( T \) using Kruskal's algorithm. Let \( W \) be its total weight.
    \item Modify edge weights:
    \begin{itemize}
        \item For each edge \( e \in T \), assign a new weight \( (w_e, 0) \).
        \item For each edge \( e \notin T \), assign a new weight \( (w_e, 1) \).
    \end{itemize}
    \item Compute a second MST \( T' \) using Kruskal's algorithm with \textbf{lexicographic order}:
    \begin{itemize}
        \item Compare edges first by original weight \( w_e \), then by the second component (0 or 1).
        \item This prioritizes edges in \( T \) when weights are equal.
    \end{itemize}
    \item Check uniqueness:
    \begin{itemize}
        \item If \( T' \) has the same original total weight \( W \) but differs from \( T \), the MST is not unique.
        \item Otherwise, it is unique.
    \end{itemize}
\end{enumerate}

\textbf{Time Complexity:} \( O(m \log m) \), dominated by sorting edges twice in Kruskal's algorithm.


\section*{Problem 4}
In this problem, we will design an algorithm to compute the minimum spanning tree in a given graph in $O(m \log \log n)$ time, where $m$ is the number of edges and $n$ is the number of vertices in the given graph. For example, if $m = \Theta(n)$, then the running time of this algorithm ($\Theta(n \log \log n)$) is better than that ($\Theta(n \log n)$) of Prim's algorithm, Kruskal's algorithm, and Bor\r{u}vka's algorithm.
Here are the ideas.
\begin{itemize}
    \item In the original Bor\r{u}vka's algorithm, we might need $\Theta(\log n)$ rounds to merge all vertices into one group, and each round takes $O(m)$ time. The twist is that we now only run Bor\r{u}vka's algorithm for $r$ rounds.
    \item After running Bor\r{u}vka's algorithm for $r$ rounds, we obtain a new graph with at most $\frac{n}{2^r}$ vertices and at most $m$ edges. We run Prim's algorithm (with a Fibonacci heap) on it to find its MST.
\end{itemize}
How should we choose the parameter $r$ so that the total running time becomes $O(m \log \log n)$?

\paragraph{\textcolor{brown}{Solution.}}
The proposed algorithm combines Boruvka's algorithm with Prim's algorithm. Run Boruvja's for r rounds, after which the number of remaining components is at most $\frac{n}{2^r}$. Each round takes $O(m)$ time, so r rounds take $O(r\cdot m)$. Then construct the contracted graph, where the vertices are the remaining components after r rounds. The edges are the original edges connecting these components (at most m). Run Prim's algorithm on the contracted graph. Using a Fibonacci heap, Prim's runs in  $O(m+klogk)$, where k is the number of vertices. Here, $k \leq \frac{n}{2^r}$, so time is $O(m+\frac{n}{2^r}log\frac{n}{2^r})$. \begin{itemize} \item The total time is the sum of Boruvka's r rounds ($O(r\cdot m)$) and Prim's on the contracted graph ($O(m+\frac{n}{2^r}log\frac{n}{2^r})$). Assuming m dominates (as $m\geq n - 1$ for connected graphs), the time simplifies to $O(r\cdot m + \frac{n}{2^r}log{n})$. \item To choose r, we set $r \cdot m + \frac{n}{2^r}log{n} = O(mloglogn)$. Assuming $m = \Theta(n)$, this becomes $r \cdot n + \frac{n}{2^r}log n = O(nloglogn)$. Dividing both sides, $r + \frac{logn}{2^r} = O(loglogn)$. \item To balance the terms, set r = loglogn: where first term is loglogn and second term is $\frac{logn}{2^{loglogn}} = \frac{logn}{logn} = 1$. Thus, $loglogn + 1 = O(loglogn)$. \item For general m, set r = loglogn. Boruvka's time is $O(mloglogn)$ and Prim's time is $O(m + \frac{n}{2^{loglogn}}log\frac{n}{2^{loglogn}})$. \item $2^{loglogn} = logn$, so $\frac{n}{logn}log(\frac{n}{logn}) \approx \frac{n}{logn}(logn - loglogn) = n(1 - \frac{loglogn}{logn}) = O(n)$. Thus, Prim's time is $O(m + n)$, which is $O(m)$ since $m\geq n - 1$. Total time is $O(mloglogn + m) = O(mloglogn)$. \item Therefore, to get runtime $O(mloglogn)$, set the number of Boruvka rounds r to loglogn. \end{itemize}


\section*{Problem 5}
We have an undirected graph $G = (V, E)$. Each edge has a positive length, and there is at most one edge between each pair of vertices. Find the length of the shortest simple cycle in $G$. Your algorithm should run in $O(n^3)$ time, where $n = |V|$. [Hint: You can modify the Floyd--Warshall algorithm. Recall that $f(k, u, v)$ is the length of the shortest path from $u$ to $v$ using vertices with indices of at most $k$. Consider a cycle where $k$ is the largest index among its vertices and $u$ and $v$ are the two neighbors of $k$ in the cycle.]

\paragraph{\textcolor{brown}{Solution.}}
Here is my solution.

