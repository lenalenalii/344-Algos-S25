\paragraph{\textcolor{red}{Acknowledgment:} Our team members are \textbf{Alexander Lin (al1655)}, \textbf{Pranav Tikkawar (pt422)}, and \textbf{Ivan Zheng (iz72)}}.


\section*{Problem 1}
There are $n$ children forming a circle. The $i$-th child currently has $a_i$ apples. Let $\bar a = \frac{\sum_{i = 1}^n a_i}{n}$ be the average number of apples that a child has, and we assume that it is an integer.

In one move, we can take one apple from a child and give it to an \emph{adjacent} child. What is the minimum number of moves we have to make to make every child have exactly $\bar a$ apples? Model this problem by a flow network with costs, and therefore show that this problem can be solved in polynomial time.

\paragraph{\textcolor{brown}{Solution.}}
Modeling a flow network, each child is a node in the network with nodes 1, 2,...,n. Each node i has a supply (if $a_i > \bar{a}$) or demand (if $a_i < \bar{a}$). Supply $s_i = a_i - \bar{a}$: 
\begin{itemize} \item If $s_i > 0:$ node i has $s_i$ apples to give away. \item If $s_i < 0:$ node i needs $|s_i|$ apples. \end{itemize} 
For each pair of adjacent children (i and (i mod n) + 1), created two directed edges:
\begin{enumerate} \item From i to (i mod n) + 1 \item From (i mod n) + 1 to i \end{enumerate}
Each edge has:
\begin{itemize} \item Capacity: infinite (or at least large enough for all apples) \item Cost: 1 per apple (because one move = moving one apple to an adjacent node) \end{itemize} 
Sending flow along an edge represents moving apples from one child to an adjacent child. The cost of the flow is the total number of moves. This is a minimum cost flow problem, which can be solved in polynomial time using algorithms like the Successive Shortest Path algorithm or the Cycle-Canceling algorithm. The minimum cost flow will give us the minimum number of moves needed to balance the apples among the children.


\section*{Problem 2}
Explicitly construct a bijection between each of the following pairs of sets and therefore show that they have the same cardinality.
\begin{enumerate}
    \item Natural numbers and perfect squares.
    \item $(0, 1)$ and $(1, +\infty)$.
    \item $(-1, 1)$ and $\mathbb{R}$.
    \item $[0, 1]$ and $[0, 1)$.
\end{enumerate}


\paragraph{\textcolor{brown}{Solution.}}
\begin{enumerate}
    \item We map every natural number $n$ to its square $n^2$. This is injective, since each natural number has a unique square, and it is surjective, as for each perfect square $p = q^2$, we can take its square root and get the natural number $|q|$ which maps to it. Therefore this mapping is bijective, and the two sets have the same cardinality.
    \item For every number $n$ in the set $(1, +\infty)$, we map it to its multiplicative inverse $\frac{1}{n}$. This is injective, as every $n$ maps to a unique number, and surjective, as for every number $p \in (0, 1)$, we can find its corresponding number $n \in (1, +\infty)$ by doing $n=\frac{1}{p}$. Thus the mapping is bijective.
    \item Similarly to the last problem, for every number $p$ in the subset $(-1, 1) \setminus 0$ of the first set $(-1, 1)$, we map it to its multiplicative inverse $\frac{1}{p}$ in the subset $\mathbb{R} \setminus 0$ of the second set $\mathbb{R}$. We additionally map $0$ in the first set to $0$ in the second set. This is injective, as each $n$ maps to its unique multiplicative inverse, except 0, which does not have a multiplicative inverse. However, we have mapped 0 in the first set to 0 in the second set, and since no number has multiplicative inverse 0, 0 uniquely maps to 0, satisfying injectiveness. This is surjective, as for every $q \in \mathbb(R)$ we can find its corresponding element in the first set by performing $\frac{1}{q}$. Again, the exception is 0, which we know maps to 0. Therefore we have a bijection.
    \item 
\end{enumerate}



\section*{Problem 3}
Review the halting problem, and prove that the following problem is also undecidable: Given a Turing machine, decide whether it always correctly decides the set cover problem. (Note that the given Turing machine may not halt on a valid input, in which case it does not correctly decide the set cover problem.)

\paragraph{\textcolor{brown}{Solution.}}
Here is my solution.


\section*{Problem 4}
Prove that the following problem is $\mathtt{NP}$-complete: Given $n$ positive integers that sum to $2W$, decide whether we can partition the integers into two parts with equal sum ($W$ each).

\paragraph{\textcolor{brown}{Solution.}}
Here is my solution.


\section*{Problem 5}
Prove that the following problem is $\mathtt{NP}$-complete: Given a weighted directed graph with $n$ vertices and $m$ edges, decide whether the graph has a simple cycle with sum of weights equal to $344$. The weight of each edge must be an integer, but can be positive, negative, or zero.

\paragraph{\textcolor{brown}{Solution.}}
Here is my solution.
