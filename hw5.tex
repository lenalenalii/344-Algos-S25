\paragraph{\textcolor{red}{Acknowledgment:} Our team members are \textbf{Alexander Lin (al1655)}, \textbf{Pranav Tikkawar (pt422)}, and \textbf{Ivan Zheng (iz72)}}.


\section*{Problem 1}
There are $n$ children forming a circle. The $i$-th child currently has $a_i$ apples. Let $\bar a = \frac{\sum_{i = 1}^n a_i}{n}$ be the average number of apples that a child has, and we assume that it is an integer.

In one move, we can take one apple from a child and give it to an \emph{adjacent} child. What is the minimum number of moves we have to make to make every child have exactly $\bar a$ apples? Model this problem by a flow network with costs, and therefore show that this problem can be solved in polynomial time.

\paragraph{\textcolor{brown}{Solution.}}
Modeling a flow network, each child is a node in the network with nodes 1, 2,...,n. Each node i has a supply (if $a_i > \bar{a}$) or demand (if $a_i < \bar{a}$). Supply $s_i = a_i - \bar{a}$: 
\begin{itemize} \item If $s_i > 0:$ node i has $s_i$ apples to give away. \item If $s_i < 0:$ node i needs $|s_i|$ apples. \end{itemize} 
For each pair of adjacent children (i and (i mod n) + 1), created two directed edges:
\begin{enumerate} \item From i to (i mod n) + 1 \item From (i mod n) + 1 to i \end{enumerate}
Each edge has:
\begin{itemize} \item Capacity: infinite (or at least large enough for all apples) \item Cost: 1 per apple (because one move = moving one apple to an adjacent node) \end{itemize} 
Sending flow along an edge represents moving apples from one child to an adjacent child. The cost of the flow is the total number of moves. This is a minimum cost flow problem, which can be solved in polynomial time using algorithms like the Successive Shortest Path algorithm or the Cycle-Canceling algorithm. The minimum cost flow will give us the minimum number of moves needed to balance the apples among the children.


\section*{Problem 2}
Explicitly construct a bijection between each of the following pairs of sets and therefore show that they have the same cardinality.
\begin{enumerate}
    \item Natural numbers and perfect squares.
    \item $(0, 1)$ and $(1, +\infty)$.
    \item $(-1, 1)$ and $\mathbb{R}$.
    \item $[0, 1]$ and $[0, 1)$.
\end{enumerate}


\paragraph{\textcolor{brown}{Solution.}}
\begin{enumerate}
    \item We map every natural number $n$ to its square $n^2$. This is injective, since each natural number has a unique square, and it is surjective, as for each perfect square $p = q^2$, we can take its square root and get the natural number $|q|$ which maps to it. Therefore this mapping is bijective, and the two sets have the same cardinality.
    \item For every number $n$ in the set $(1, +\infty)$, we map it to its multiplicative inverse $\frac{1}{n}$. This is injective, as every $n$ maps to a unique number, and surjective, as for every number $p \in (0, 1)$, we can find its corresponding number $n \in (1, +\infty)$ by doing $n=\frac{1}{p}$. Thus the mapping is bijective.
    \item Similarly to the last problem, for every number $p$ in the subset $(-1, 1) \setminus 0$ of the first set $(-1, 1)$, we map it to its multiplicative inverse $\frac{1}{p}$ in the subset $\mathbb{R} \setminus 0$ of the second set $\mathbb{R}$. We additionally map $0$ in the first set to $0$ in the second set. This is injective, as each $n$ maps to its unique multiplicative inverse, except 0, which does not have a multiplicative inverse. However, we have mapped 0 in the first set to 0 in the second set, and since no number has multiplicative inverse 0, 0 uniquely maps to 0, satisfying injectiveness. This is surjective, as for every $q \in \mathbb(R)$ we can find its corresponding element in the first set by performing $\frac{1}{q}$. Again, the exception is 0, which we know maps to 0. Therefore we have a bijection.
    \item Map every number from the first set to the same number in the second set, except for numbers that can be expressed in the form $\frac{1}{n}$, where $n$ is an integer greater than 0, which we instead map to $\frac{1}{n+1}$. For example, $1=\frac{1}{1}$ in the first set maps to $\frac{1}{2}$ in the second, $\frac{1}{2}$ in the first set maps to $\frac{1}{3}$ in the second, and so on. This means no number in the first set maps to $1$ in the second set. This is injective, as each number $a$ in the first set either cannot be written in the form $\frac{1}{n}$ for some natural number $n$, in which case it maps to itself uniquely, or it can be written in that form, in which case it maps uninquely to $\frac{1}{n+1}$. This is injective, as for a number $b$ in the second set that does not follow the form $\frac{1}{n}$, it will be mapped to by themselves, and for $b$ that does follow the form, we can find the number that maps to it as $\frac{1}{n-1}$. Therefore the mapping is bijective.
\end{enumerate}



\section*{Problem 3}
Review the halting problem, and prove that the following problem is also undecidable: Given a Turing machine, decide whether it always correctly decides the set cover problem. (Note that the given Turing machine may not halt on a valid input, in which case it does not correctly decide the set cover problem.)

\paragraph{\textcolor{brown}{Solution.}}
Here is my solution.


\section*{Problem 4}
Prove that the following problem is $\mathtt{NP}$-complete: Given $n$ positive integers that sum to $2W$, decide whether we can partition the integers into two parts with equal sum ($W$ each).

\paragraph{\textcolor{brown}{Solution.}}
Here is my solution.


\section*{Problem 5}
Prove that the following problem is $\mathtt{NP}$-complete: Given a weighted directed graph with $n$ vertices and $m$ edges, decide whether the graph has a simple cycle with sum of weights equal to $344$. The weight of each edge must be an integer, but can be positive, negative, or zero.

\paragraph{\textcolor{brown}{Solution.}}
To prove a problem is NP-complete, we first need to show the problem is in NP. To do this, we can check that given a cycle as a certificate, we can check if it's simple and that the sum of edge weights along the cycle is exactly 344. Both of these can be done in polynomial time, thus the problem is in NP. 
\\ \\ Next, to show that the problem is NP-hard, we need to reduce a known NP-complete problem to it. The Subset Sum problem says that: 
\begin{itemize} \item given integers $a_1, a_2, ..., a_n$ and integer K does there exist a subset whose sum is exactly K? \end{itemize}
We can reduce the Subset Sum problem (which is known NP-complete) to our cycle problem. We start by building the graph:
\begin{enumerate} \item Create n + 1 vertices $v_0, v_1, v_2, ..., v_n$. \item For each i = 1, ...,n: \begin{itemize} \item Add a directed edge from $v_{i-1}$ to $v_i$ with weight 0. \item Add a directed edge from $v_{i-1}$ to $v_i$ with weight $a_i$. \end{itemize} \item Add a directed edge from $v_n$ to $v_0$ with weight 0. \end{enumerate}
Next is the cycle correspondence:
\begin{enumerate} \item Any simple cycle starting at $v_0$, going through to $v_n$, and back to $v_0$ corresponds to a selection of edges. \item At each step i, you choose either: \begin{itemize} \item The 0-weight edge (meaning don't include $a_i$), or \item The $a_i$-weight edge (meaning include $a_i$ in the sum). \end{itemize} \end{enumerate}
Thus, the total weight of the cycle is the sum of the selected $a_i$ values.
\\ The next step is to set the target weight:
\begin{enumerate} \item Set the target weight to K. \item The cycle has weight K $\iff$ there is a subset of the $a_i$ summing to K. \end{enumerate}
Finally, we map to 344:
\begin{itemize} \item Since the cycle problem has a fixed target of 344, we can transform any instance of Subset Sum to an instance of our problem by shifting the target appropriately (e.g., for any Subset Sum instance, ask whether there is a cycle of weight $K' = 344$ by adjusting edge weights by $(344 - K)$ if needed). \end{itemize}
Therefore, the problem is NP-complete.