\paragraph{\textcolor{red}{Acknowledgment (remember to edit this).}} Our team members are \textbf{Isaac Newton (in123)}, \textbf{Albert Einstein (ae123)}, and \textbf{Alan Turing (at123)}. We searched for Problem 17 on the Internet after spending 123 hours on it, and we asked ChatGPT how to solve Problem 32(c) after spending 123 days on it. We did not successfully solve Problem 49 despite having spent 123 years on it. \textbf{We affirm that we have acknowledged every external help during the preparation of this document.}


\section*{Problem 1}
There are $n$ children forming a circle. The $i$-th child currently has $a_i$ apples. Let $\bar a = \frac{\sum_{i = 1}^n a_i}{n}$ be the average number of apples that a child has, and we assume that it is an integer.

In one move, we can take one apple from a child and give it to an \emph{adjacent} child. What is the minimum number of moves we have to make to make every child have exactly $\bar a$ apples? Model this problem by a flow network with costs, and therefore show that this problem can be solved in polynomial time.

\paragraph{\textcolor{brown}{Solution.}}
Here is my solution.


\section*{Problem 2}
Explicitly construct a bijection between each of the following pairs of sets and therefore show that they have the same cardinality.
\begin{enumerate}
    \item Natural numbers and perfect squares.
    \item $(0, 1)$ and $(1, +\infty)$.
    \item $(-1, 1)$ and $\mathbb{R}$.
    \item $[0, 1]$ and $[0, 1)$.
\end{enumerate}


\paragraph{\textcolor{brown}{Solution.}}
\begin{enumerate}
    \item Here is my solution.
    \item 
    \item 
    \item 
\end{enumerate}



\section*{Problem 3}
Review the halting problem, and prove that the following problem is also undecidable: Given a Turing machine, decide whether it always correctly decides the set cover problem. (Note that the given Turing machine may not halt on a valid input, in which case it does not correctly decide the set cover problem.)

\paragraph{\textcolor{brown}{Solution.}}
Here is my solution.


\section*{Problem 4}
Prove that the following problem is $\mathtt{NP}$-complete: Given $n$ positive integers that sum to $2W$, decide whether we can partition the integers into two parts with equal sum ($W$ each).

\paragraph{\textcolor{brown}{Solution.}}
Here is my solution.


\section*{Problem 5}
Prove that the following problem is $\mathtt{NP}$-complete: Given a weighted directed graph with $n$ vertices and $m$ edges, decide whether the graph has a simple cycle with sum of weights equal to $344$. The weight of each edge must be an integer, but can be positive, negative, or zero.

\paragraph{\textcolor{brown}{Solution.}}
Here is my solution.
